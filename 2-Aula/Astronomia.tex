Após vermos o inicio do pensamento astronômico, agora devemos ver aqueles que primordialmente formaram algumas das bases da astronomia ocidental e que foi difundida por milênios. Produzindo materiais para estudo, e formulando teses e conceitos que seriam seguidos e contestados apenas séculos posteriormente devido a importância e credibilidade dos pensadores helênicos.

\begin{figure}[ht!]
\includegraphics[scale= 0.3]{gregos}
\end{figure}
\section*{Cosmologia Grega}
Os habitantes do império grego foram grandes devotos dos astros, estudando e descrevendo a esfera celeste com grande afinco e precisão para os instrumentos disponíveis para a época.

Foram eles quem definiram e nomearam grande parte do céu da forma que conhecemos atualmente, dando os nomes e formas para as constelações e divisões celestes utilizadas pelos povos ocidentais por milênios. Usando os seus deuses e mitos para nomear e classificar o céu, e também eram estes os motivos e explicações dos acontecimentos naturais. Porém no século 6 a.C. surgem aqueles que queriam modificar e repensar isto, os pensadores pré-socráticos, que buscavam explicações lógicas e não mitológicas para os eventos das natureza, como as chuvas, raios, os elementos que compunham tudo, a existência dos seres, os movimentos celestes e muitos outros efeitos naturais.
\section*{Jônicos}
Os Jônicos foram os primeiros a fazerem e buscarem respostas científicas para algumas das grandes perguntas que existiam e ainda existem, como, do que somos feitos? De onde vem todas as coisas?, além de tentarem descrever os fenômenos da natureza utilizando a matemática conhecida. Dentre esses pensadores, se destacaram alguns nascidos na cidade de Miletus.
\subsection*{Thales(625-547 a.C.)}
Um dos mais conhecidos popularmente é o pensador Thales de Miletus,sendo o primeiro filósofo ocidental conhecido, que realizou grandes contribuições para a geometria, astronomia, filosofia e diversas áreas do conhecimento.

Thales, foi o primeiro a descrever como ocorria um eclipse solar, e também previu um eclipse com certa precisão. Mas o grande interesse de Thales era saber e pensar sobre a origem fundamental de tudo, além de procurar formas racionais de descrever a Physis. Para Thales, o elemento fundamental e originador de todos os seres vivos era a água, todos os seres teriam evoluídos de corpos d'água, uma ideia quase que Darwiniana 6 séculos antes de cristo, e tudo isto devido as suas observações com plantas e animais, pois os alimentos sempre contêm umidade, assim como para uma planta germinar é necessário que ela seja regada com frequência e os animais para se originarem necessitam do sêmen.

Posteriormente a Thales, vieram outros jônicos seguindo linhas de pensamento parecidas, ou tentando responder as mesmas perguntas de modos diferentes, e isto é um dos primeiros formatos do que chamamos de ciência atualmente, onde modelos pode ser propostos, repensados, descartados e debatidos, assim gerando um progresso.
\subsection*{Anaximandro(610-546 a.C.)}
Aluno de Thales, Anaximandro de Miletus também se questionava sobre os a origens das coias e como elas poderiam ser descritas ou explicadas sem a utilização de entidades mitológicas, mas diferentemente de seu mestre, Anaximandro procurou saber sobre a origem fundamental de tudo, do universo e de todo o seu entorno, assim formulando uma nova teoria de forma e origem cosmológica.

Para Anaximandro, no princípio havia calor e frio e estes corpos começaram a se separar e então o calor estava rodeado por uma neblina, então a bola de fogo diminuiu e se tornou uma massa sólida que era a Terra, porém alguns anéis de neblina ficaram presos dentro da massa de calor, se tornando a nossa atmosfera, e por buracos nesta neblina conseguíamos observar os outros corpos de fogo, como a Lua, o Sol, e as estrelas, que foram também aprisionados e reduzidos pelas massas de ar gelados.

Sendo discípulo de Thales, Anaximandro também estava tentando responder qual era o elemento fundamental e formador de todos os outros, e para ele, diferente de seu mestre, o "Ilimitado" era o elemento fundamental, a massa primordial que separou o calor do frio no inicio dos tempos, sendo eterno e imutável.

\subsection*{Anaximenes(585-525 a.C.)}
\begin{figure}[!htb]
	\centering
	\includegraphics[scale= 1.0]{anaximenes.jpg}
	
\end{figure}
Outro morador e estudioso de Miletus, Anaximenes era também pensador da escola Jônica e buscava assim como seus conterrâneos um elemento fundamental e originário de todo o resto.

Para ele, este elemento era o ar,pois Anaximenes já tinha observado que sem o ar os seres morrem e não se desenvolvem, entretanto, o material fundamental não este ar que nos cerca unicamente, mas sim um ar puro, primordial, imutável e que estava em toda nossa volta. O ar infinito ou {\it pneuma apeiron}, todos os seres teriam sido originários deste ar, e ele estaria dentro de cada um, sendo a nossa alma, ou o ar interno de cada pessoa, que a trazia a vida e nos mantinha unidos.

E para Anaximenes, o {\it pneuma apeiron} também estava envolvido com o processo de formação e funcionamento da terra, pois todos os elementos eram formados de correntes de condensação e rarefação deste ar, assim formando os corpos sólidos. E a Terra seria plana assim como os outros corpos celestes e estariam todos estes flutuando no grande ar infinito como folhas que caem de uma arvore.

\section*{Pitagóricos}
\subsection*{Pitágoras(570-495 a.C.)}
\begin{figure}[!htb]
	\centering
	\includegraphics[scale= 0.8]{pitagoras.png}
\end{figure}
Um dos mais famosos e lembrados pensadores gregos foi Pitágoras de Samos, por suas contribuições para matemática, geometria, música e filosofia, fundador de uma escola que misturava os pensamentos naturais com misticismo, assim sendo reconhecido por alguns como um culto também. Porém, não se sabe exatamente o que foi realmente produzido unicamente por Pitágoras, pois os alunos, ou devotos, de sua escola eram obrigados a ter um anonimato quanto aos seus feitos pessoais, assim o único que levava o crédito era Pitágoras.

Dentre os mais importantes trabalhos de Pitágoras está o Teorema de Pitágoras, que já era conhecido desde os povos da Babilônia, porém o grego foi o primeiro a demonstrar e provar o teorema, assim levando toda a gloria. Outros estudos importantes estão ligados a numerologia e a música, dois assuntos de extremo interesse de Pitágoras.

Mas não só de números vivia o pensador grego helênico, ele assim como todos os seus colegas, era grande fã e estudioso do celeste, sendo o primeiro a classificar os planetas e a dizer que a estrela de manhã é Vênus. Pitágoras também foi o primeiro a supor que a Terra era esférica, pois esta era a forma perfeita e assim todos os corpos celestes deveriam ser, e complementar a isto, todos os astros conhecidos rodavam em volta da Terra que também estava em rotação.

Sendo um grande estudioso da música, Pitágoras tinha a teoria de que existia uma "Música das Esferas", onde cada corpo teria associado a si uma nota baseada nas proporções matemáticas encontradas, e que nós não escutamos esta música pois já nascemos com ela tocando, então estamos acostumados com os sons celestiais.

\subsection*{Filolau(470-385 a.C.)}
\begin{figure}[!htb]
	\centering
	\includegraphics[scale= 0.3]{filolau.png}
	\includegraphics[scale= 0.1]{geocentrismo.png}
\end{figure}

Filolau de Crotona foi aluno da escola pitagórica, então assim como seu mestre também buscava a ordem e perfeição numérica nos acontecimentos da natureza.

Para isto, Filolau sugeriu a existência de um fogo central, que estava no centro do universo e que dava energia e movimento para todos os corpos celestes do universo. E para fechar a conta havia também uma anti-terra que estava mais próxima de "Héstia", o fogo central, e que seria devido a esta anti-terra que nunca vemos e nem somos incinerados pelo fogo central, e juntamente a isto surge a necessidade de inserir uma rotação na Terra para que o efeito de dia e noite ocorra, já que não estamos mais no centro de tudo e sim girando em torno de um objeto mais importante, assim sendo teriam 10 corpos celestes no universo gerando a harmonia necessária para tudo.

Com o acréscimo destes dois corpos, Filolau foi o primeiro a remover aTerra, e assim os gregos, do centro do universo, e portanto não haveria algo mais importante que eles, e com isto foi dado aos deuses Héstia.







