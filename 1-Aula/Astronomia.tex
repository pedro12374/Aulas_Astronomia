
\section{O que é Astronomia?}
Pelo google astronomia é:
\begin{quote}
Astronomia($\alpha \sigma \tau \rho o\upsilon $(ástron,"estrelas")+$\upsilon \acute{o} \mu o \zeta$(nómos,"leis"))\\
Ciência que trata do universo sideral e dos corpos celestes, com o fim de situá-los no espaço e no tempo e explicar sua origem e seu movimento.
\end{quote}
Durante todo esse período, vamos estudar o celeste, suas regras e leis, e também sua história e características. Além dos aspectos cosmológicos que cercam a Terra e a humanidade durante toda a nossa existência.
 
Abordando assuntos como, formação do Sistema Solar, leis físicas de movimento e atração, morfologia dos planetas próximos, foguetes e satélites, astrologia e cosmologia, entre muitos outros assuntos que abordam as ideias já citadas.
\section{Origem da Astronomia}
\begin{quote}
{\large\it 'Quando os hominídeos primitivos largaram sua postura curva e começaram a ficar eretos, eles não queriam se tornar o que hoje é conhecido como homo sapiens, na verdade, eles só buscavam por um breve vislumbre dos céus'}   {\bf Danilo Reynan}
\end{quote}
O interesse pelo celeste fez parte do interesse humano, estando presente em quase todas as civilizações conhecidas, e gerando interesse na humanidade desde tempos imemoriais.

Diversas culturas aprenderam a utilizar o celeste para guiar como as coisas seriam feitas na Terra, representando os astros como Deuses, e os eventos astronômicos que ocorriam sendo manifestações deles. E percebendo a regularidade de certos eventos e seu impacto com o que ocorria nos meios terrenos, então se percebeu a importância de estudar o celeste. Para assim poder dizer, quando era o momento de plantar, colher, trabalhar, como seria o verão e o inverno, entre muitos outros fatores que eram possíveis de se determinar apenas estudando o céu.
\subsection*{Pré-Historia}
Os mais antigos objetos de estudo astronômico que se tem noticias são os megálitos. Grandes estruturas com um ou mais blocos de pedra gigantes, um dos exemplos mais conhecido é o stonehenge, que foi construído em aproximadamente 3100 a.C, onde segundo pesquisadores, ocorriam grandes festivais para comemorar os solstícios e equinócios.

 De dentro da grande estrutura, de acordo com seu posicionamento em um dos 70 círculos de auxilio que ele possuía era possível determinar quando ocorreria a troca de estações e assim obter uma melhor produção agrícola, alem de saber quanto tempo de preparação para o inverno teriam, entre outros fatores que o cosmos entregava.

\begin{figure}[!h]
	\center
	\subfigure[Stonehenge]{\includegraphics[scale= 0.3]{Stn}}
	\qquad
	\subfigure[Grand Menhir Brisée]{\includegraphics[scale= 0.45]{Mgl}}
\end{figure}
Este período é marcado pela dita "cosmologia magica", pois todos os eventos que ocorriam fora do seu entorno, ou seja, no celeste, eram causados pela vontade divina, e todos os elementos externos tinham vida própria, consciência e vontade própria, e para que esses seres não prejudicassem a vida no plano mortal, eram realizadas oferendas e sacrifícios em nome dos seres divinos.

Porém com a formação de sociedades mais estruturadas e as primeiros pensamentos lógicos sendo formados, tem-se o começo de uma era onde os eventos ocorridos eram ainda atribuídos e historias e mitos, porem dentro destes mitos haviam muito mais explicações lógicas dos ocorridos, e não apenas ideias internalizadas de representações humanas, e esta fase é classificada como "Cosmologia Mítica".
\subsection*{Mesopotâmios}
Após a invenção da escrita pelos povos antigos, tem-se o fim da pré-história e o inicio da era dos metais.

Os povos que habitaram a região da crescente fértil, fundaram e estabeleceram diversos conceitos astronômicos que são utilizados até hoje, além de começarem a formalização do estudo das esferas celestes.Mas quando falamos de mesopotâmios, não estamos nos referindo a um povo ou cultura, e sim a uma localização povos que habitaram este local físico, situado na região dos rios Tigre e Eufrates, a mesopotâmia se estende por grandes planícies férteis, onde grande povos se desenvolveram e findaram.
\subsubsection*{Babilônios}
Os povos que habitaram a antiga babilônia foram os primeiros a deixarem registros lógicos e matemáticos de seus estudos, com placas demonstrando a sua matemática e forma de pensamento.

Grade parte do povo que viviam na região da babilônia erma sumérios, os detentores de uma cultura milenar e que desenvolveram um dos mais antigos sistemas de escrita conhecidos em suas tabuas de escrita cuneiforme. Entretanto o conhecimento nesta época era restrito apenas a profetas e sacerdotes, assim, pouco era desenvolvido caso não houvesse interesse real, ou do próprio estudioso. Mas para sorte dos arqueólogos, os babilônios fizeram inúmeros registros sobre as estrelas, planetas, as fazes e movimentos da lua, além de outros registros astronômicos, todos com muita precisão e detalhes para época, além das tabelas em que era possível prever o movimento dos astros e assim dizer sobre a fertilidade e sucesso do reino, estas anotações são os registros científicos mais antigos já encontrados, com mais de 2800 anos de idade.
\subsubsection*{Egípcios}
Vizinhos não muito distantes dos babilônios, os povos que habitaram o antigo Egito eram muito menos interessados no celeste que seus vizinhos jardineiros. Estes focaram no desenvolvimento em outras áreas, como arquitetura, literatura, artes e matemática, e acabaram por não ter um foco tão grande nos objetos do cosmos, somente foram desenvolver interesse por esta área em 300a.C, quase 2700 anos após o inicio do império egípcio.

Contudo, os egípcios estabeleceram o sistema atual de divisão do dia e noite em 12 horas, e também estabeleceram o ano como 12 meses e 365 dias. Mas grande parte destes fatos é devido a uma popularização e influência das crenças presentes na região, assim espalhando mais a religião egípcia.
\subsubsection*{Astrologia}
Muito já foi dito sobre a associação mundana com o celeste de forma mistica e mágica, e uma das vertentes que se mantem popular até hoje e que faz parte dos nossos estudos é a astrologia(ástron,"astros" + logos,"estudo"), que so foi diferenciada da astronomia científica a pouquíssimos séculos.

A astrologia popularmente estuda os efeitos que os corpos celestes exercem sobre a terra e no nascimento das pessoas, relacionando comportamentos e atitudes a signos e constelações do zodíaco, mas isto não surgiu simplesmente do nada. O inicio da astronomia se deu com a divisão e o estudo do céu em 12 partes de 30º cada, totalizando os 360º de rotação da terra, e cada parte teria associado uma constelação em uma faixa do céu, o chamado Zodíaco, e durante todo a constelação pela qual o sol cruzava o céu era a regente. A primeira e em que se iniciava efetivamente o ano era a constelação de Áries, ou o carneiro, que se dá no inicio da primavera no hemisfério norte, nas datas de 21 de março até 20 de abril, neste período o inverno tinha-se findado e era chegado o momento de tosar os carneiros e vender a lã para movimentar o mercado, então todo este inicio de estação era voltado aos caprinos, e enquanto isto no celeste o sol passava pela constelação de Aries, então com o tempo foi criada a associação dos esteriótipos com a época do ano e a constelação regente no céu.

\begin{figure}[h!]
\centering
\subfigure{\includegraphics[scale=0.6]{Zodiac}} 
\subfigure{\includegraphics[scale=0.25]{Zodiac1}}

\caption{Panorama geral do Zodíaco}
\end{figure}

Mas devido ao movimento de precessão da Terra, muitas das constelações não tem mais os seus 30º, surgiram novos signos, e outros estão para desaparecer.
\subsection*{Ásia}
Um dos maiores erros que se pode cometer é achar que somente as civilizações europeias tiveram grandes avanços e estudos em todas as áreas do conhecimento, quando o oriente foi o responsável por inúmeras invenções e descobertas, que mais tarde os europeus iriam refinar ou contradizer e devido a sua influencia global serem aceitos como os corretos. Então vamos ver um pouco da astronomia antiga desenvolvida pelos povos orientais.
\subsubsection*{Índia}
Os povos que compõem a Índia possuem uma das mais antigas e bem preservadas historias, com grande parte dos acontecimentos registrados nos Vedas sagrados. O mais antigo destes textos é o Rig Veda, o livro sagrado dos hindus, com histórias e relatos anteriores a 1900 a.C, e segundo as historias tradicionais, o Rig Veda é anterior a 3100 a.C.

Tendo tantos anos de historia e cultura, dentro dos Vedas há também diversas anotações e relatos sobre eventos astronômicos que ocorreram, e as suas influencias na vida terrestre. Mesmo estando junto aos textos sagrados, os Vedas são recheados de lógica em suas escrituras, sempre procurando explicações funcionais e que façam sentido, e não apenas soluções por deuses. 

Entretanto, a religião Hindu acaba por deter diversos conceitos modernos de pensamento de universo, como a infinidade do universo no espaço e no tempo, sendo representado pela dança da deusa Shiva, a ciclicidade do universo é algo que já está incorporado na filosofia e pensamento Hindu, enquanto Shiva dança pelo "kalpa", o universo é destruído e recriado, e cada kalpa tem a duração de um dia para Brahma, ou 4,32 bilhões de anos para nós mortais.


Outros conceitos cientifico que os Hindus já desenvolvem e pregam a milênios é a de relatividade do espaço e tempo, onde em certas historias, o tempo passa de forma diferente para observadores diferentes, uma pré ideia do que seria formulado por Einstein em 1905, mas deve-se lembrar que estas historias não tem nenhuma ligação com a velocidade da luze afins.


Um ponto que deve ser sempre lembrado é que os indianos sempre tiveram muito interesse em descrever de forma matemática os movimentos realizados pelos astros, assim surge Aryabhata, um dos cientistas Hindus que brigava de frente com os conceitos que os gregos espalhavam pelo mundo, sendo contrário a ideia de Terra estacionaria, e aplicando formulas e equações geométricas e algébricas no celeste, Aryabhata estava preocupado em utilizar um método realmente matemático para a explicações da mecânica celeste e não se contentava com as explicações sempre faltantes de Ptolomeu.

\begin{figure}[!h]
	\center
	\subfigure[Shiva]{\includegraphics[scale= 0.2]{Shiva}}
	\qquad
	\subfigure[Mapa celeste chines]{\includegraphics[scale= 0.9]{Chin}}
\end{figure}



\subsubsection*{China}
Na terra do dragão, o interesse pela astronomia também possui milhares de anos já, com anotações dos seculos 3 e 4 antes de Cristo, tendo feito as suas próprias constelações(bem diferente das que conhecemos, já que não existe motivo físico para os desenhos).

Assim como os outros povos, todos os aspectos da cultura eram influenciados e levavam símbolos e ideias astronômicos, desde a arquitetura, arte, literatura e também no Taoismo e Confucionismo, as duas maios religiões chinesas.

Tendo os céus como um interesse, os chineses desenvolveram diversas teorias a cerca do funcionamento do cosmos, já distinguindo os planetas e estrelas e também os movimentos errantes dos corpos celestes. Assim conseguindo desenvolver vários modos de pensamento, sendo os principais {\it Gai Tian},{\it Hun Tian} e {\it Xuan Ye}. A primeira descreve o mundo com um firmamento, centralizada em cima a ursa maior e na terra a china, para os pensadores do Hun Tian o firmamento possuía uma forma ovalada e a Terra seria a gema deste grande ovo, sendo mantida suspensa no meio do ovo por um vapor cósmico chamado 'qi'.

Um ponto interessante a ser notado é a mostra que os chineses já aceitavam e imaginavam o movimento celeste como sendo de uma forma não circular, algo que levaria quase 1000 anos para ocorrer no ocidente.

E para a escola de Xuan Ye, a Terra assim como todos os outros corpos celestes estavam suspensos em um universo infinito e vazio. Devido a sua localização geográfica a cosmologia chinesa tem muitos fatores que são muito diferentes do que estamos acostumados, para eles o centro do céu era a chamada atualmente de ursa maior, com a estrela polar, e lá seria o polo norte do planeta, logo abaixo estaria a china e quanto mais afastado do centro do mundo, mais os povos eram selvagens, pois estariam mais distantes do ponto máximo do céu.


